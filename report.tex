\documentclass{article}
\usepackage[utf8]{inputenc}
\title{COMP 562 Final Project: Fake or Real News Detection}
\author{Ziqian Xu, Yinuo Hu, Mingli Zhang, Xinran Ming}
\date{}
\begin{document}
\maketitle
\section{Introduction}
Fake news are gaining attention nowadays both due to the increasing use of internet and medias and due to the rise in political, social, and cultural awareness. News sources such as \emph{The Onion} which intentionally publish fake information are also popular (Brodie, 2018). Thus, it has become more and more important to identify the truthfulness of news.

Previous studies have attempted to use machine learning to distinguish real news and fake news. Researchers have attempted to classify news using Logistic Regression, Naive Bayes, Support Vector Machine, and decision trees (Katsaros et al., 2019; Granik et al., 2017). Different types of artificial neural networks including Long Short-Term Memory Recurrent Neural Networks, Convolutional Neural Networks, and Adversarial Neural Networks were also widely used in news detection (Bahad et al., 2019; Wang et al., 2018; Yang et al., 2018). 

Because of the plethora of algorithms available to classify fake and real news, the current study aims to compare the accuracy of [algorithm 1], [algorithm 2], [algorithm 3], [algorithm 4] using the [F1? AUC? other accuracy criteria] as a criterion. 

\section{Methods}
\subsection{Data}
We obtained two datasets, one with all fake news, and one with all real news, from https://www.kaggle.com/clmentbisaillon/fake-and-real-news-dataset. 
\subsection{Preprocessing}
\subsection{Models}

\section{Results}

\Section{Discussions}

\section{References}

\quad \, Ahmed, H., Traore, I., \& Saad, S. (2018). Detecting opinion spams and fake news using text classification. Security and Privacy, 1(1), e9.

Bahad, P., Saxena, P., \& Kamal, R. (2019). Fake News Detection using Bi-directional LSTM-Recurrent Neural Network. Procedia Computer Science, 165, 74-82.

Brodie, I. (2018). Pretend news, false news, fake news: The onion as put-on, prank, and legend. Journal of American Folklore, 131(522), 451-459.

Granik, M., & Mesyura, V. (2017, May). Fake news detection using naive Bayes classifier. In 2017 IEEE First Ukraine Conference on Electrical and Computer Engineering (UKRCON) (pp. 900-903). IEEE.

Katsaros, D., Stavropoulos, G., & Papakostas, D. (2019, October). Which machine learning paradigm for fake news detection?. In 2019 IEEE/WIC/ACM International Conference on Web Intelligence (WI) (pp. 383-387). IEEE.

Wang, Y., Ma, F., Jin, Z., Yuan, Y., Xun, G., Jha, K., ... \& Gao, J. (2018, July). Eann: Event adversarial neural networks for multi-modal fake news detection. In Proceedings of the 24th acm sigkdd international conference on knowledge discovery & data mining (pp. 849-857).

Yang, Y., Zheng, L., Zhang, J., Cui, Q., Li, Z., \& Yu, P. S. (2018). TI-CNN: Convolutional neural networks for fake news detection. arXiv preprint arXiv:1806.00749.


\end{document}
